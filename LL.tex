\documentclass[english]{lipics-stripped}
% \usepackage[utf8]{inputenc}
\usepackage{mathpartir}
% \usepackage{amsmath}
% \usepackage{amsthm}
% \usepackage{amssymb}
\usepackage{bussproofs}
\usepackage{ cmll }
\EnableBpAbbreviations

\title{An ISWIM syntax for linear-logic sequents}
\author{Jean-Philippe Bernardy}
\affil{Chalmers University of Technology and University of Gothenburg\\
  \texttt{bernardy@chalmers.se}}
 \authorrunning{J-P. Bernardy}%optional

\Copyright[nc-sa]%choose "nd" or "nc-nd"
          {Jean-Philippe Bernardy}

\DeclareUnicodeCharacter{2294}{\sqcup} % ⊔
\DeclareUnicodeCharacter{2293}{\sqcap} % ⊓
\DeclareUnicodeCharacter{00A0}{~} %  
\DeclareUnicodeCharacter{22EE}{\ensuremath{\vdots}} % ⋮
\DeclareUnicodeCharacter{223C}{\ensuremath{\sim}} % ∼
\DeclareUnicodeCharacter{00AC}{\ensuremath{\neg}} % ¬
\DeclareUnicodeCharacter{00B0}{^{\circ}} % °
\DeclareUnicodeCharacter{00B1}{^1} 
\DeclareUnicodeCharacter{00B2}{^2} % ²
\DeclareUnicodeCharacter{00B7}{\ensuremath{\cdot}} % ·
\DeclareUnicodeCharacter{00B9}{\textsuperscript{l}} % ¹
\DeclareUnicodeCharacter{00D7}{\ensuremath{\times}} % × 
\DeclareUnicodeCharacter{00F7}{\ensuremath{\div}} % ÷
\DeclareUnicodeCharacter{0393}{\ensuremath{\Gamma}} % Γ
\DeclareUnicodeCharacter{0394}{\ensuremath{\Delta}} % Δ
\DeclareUnicodeCharacter{0398}{\ensuremath{\Theta}} % Θ
\DeclareUnicodeCharacter{039B}{\ensuremath{\Lambda}} % Λ
\DeclareUnicodeCharacter{039E}{\ensuremath{\Xi}} % Ξ
\DeclareUnicodeCharacter{03A0}{\ensuremath{\Pi}} % Π
\DeclareUnicodeCharacter{03A3}{\ensuremath{\Sigma}} % Σ
\DeclareUnicodeCharacter{03B1}{\ensuremath{\mathnormal{\alpha}}} % α
\DeclareUnicodeCharacter{03B2}{\ensuremath{\beta}} % β
\DeclareUnicodeCharacter{03B3}{\ensuremath{\mathnormal{\gamma}}} % γ
\DeclareUnicodeCharacter{03B4}{\ensuremath{\mathnormal{\delta}}} % δ
\DeclareUnicodeCharacter{03B5}{\ensuremath{\mathnormal{\varepsilon}}} % ε
\DeclareUnicodeCharacter{03B7}{\ensuremath{\mathnormal{\eta}}} % η
\DeclareUnicodeCharacter{03B8}{\ensuremath{\mathnormal{\theta}}} % θ
\DeclareUnicodeCharacter{03B9}{\ensuremath{\mathnormal{\iota}}} % ι
\DeclareUnicodeCharacter{03BA}{\ensuremath{\mathnormal{\kappa}}} % κ
\DeclareUnicodeCharacter{03BB}{\ensuremath{\mathnormal{\lambda}}} % λ
\DeclareUnicodeCharacter{03BC}{\ensuremath{\mathnormal{\mu}}} % μ
\DeclareUnicodeCharacter{03BD}{\ensuremath{\mathnormal{\mu}}} % ν
\DeclareUnicodeCharacter{03BE}{\ensuremath{\mathnormal{\xi}}} % ξ
\DeclareUnicodeCharacter{03C0}{\ensuremath{\mathnormal{\pi}}} % π
\DeclareUnicodeCharacter{03C1}{\ensuremath{\mathnormal{\rho}}} % ρ
\DeclareUnicodeCharacter{03C3}{\ensuremath{\mathnormal{\sigma}}} % σ
\DeclareUnicodeCharacter{03C4}{\ensuremath{\mathnormal{\tau}}} % τ
\DeclareUnicodeCharacter{03C6}{\ensuremath{\mathnormal{\varphi}}} % φ
\DeclareUnicodeCharacter{03C7}{\ensuremath{\mathnormal{\chi}}} % χ
\DeclareUnicodeCharacter{03C8}{\ensuremath{\mathnormal{\psi}}} % ψ
\DeclareUnicodeCharacter{03C9}{\ensuremath{\mathnormal{\omega}}} % ω 
\DeclareUnicodeCharacter{03F5}{\ensuremath{\mathnormal{\epsilon}}} % ϵ
\DeclareUnicodeCharacter{10627}{\ensuremath{\lbana}} 
\DeclareUnicodeCharacter{10628}{\ensuremath{\rbana}} 
\DeclareUnicodeCharacter{2026}{\ensuremath{\ldots}}
\DeclareUnicodeCharacter{202F}{{\,}}
\DeclareUnicodeCharacter{2080}{_0} % ₀
\DeclareUnicodeCharacter{2081}{_1} 
\DeclareUnicodeCharacter{2082}{_2} 
\DeclareUnicodeCharacter{2115}{\mathbb{N}}
\DeclareUnicodeCharacter{2190}{\ensuremath{\leftarrow}} % ← 
\DeclareUnicodeCharacter{2191}{\ensuremath{\uparrow}} % ↑
\DeclareUnicodeCharacter{2192}{\ensuremath{\rightarrow}} % →
\DeclareUnicodeCharacter{2194}{\ensuremath{\leftrightarrow}} % ↔
\DeclareUnicodeCharacter{2196}{\nwarrow} % ↖
\DeclareUnicodeCharacter{2197}{\nearrow} % ↗
\DeclareUnicodeCharacter{219D}{\ensuremath{\leadsto}} % ↝
\DeclareUnicodeCharacter{21A6}{\ensuremath{\mapsto}} % ↦ 
\DeclareUnicodeCharacter{21C6}{\ensuremath{\leftrightarrows}} % ⇆
\DeclareUnicodeCharacter{21D0}{\ensuremath{\Leftarrow}} % ⇐
\DeclareUnicodeCharacter{21D2}{\ensuremath{\Rightarrow}} % ⇒ 
\DeclareUnicodeCharacter{21D4}{\ensuremath{\Leftrightarrow}} % ⇔
\DeclareUnicodeCharacter{2200}{\ensuremath{\forall}} % ∀
\DeclareUnicodeCharacter{2203}{\ensuremath{\exists}} % ∃
\DeclareUnicodeCharacter{2205}{\ensuremath{\varnothing}} % ∅
\DeclareUnicodeCharacter{2208}{\ensuremath{\in}} % ∈
\DeclareUnicodeCharacter{2209}{\ensuremath{\not\in}} % ∉
\DeclareUnicodeCharacter{220B}{\ensuremath{\ni}}
\DeclareUnicodeCharacter{2211}{\sum}% ∑
\DeclareUnicodeCharacter{2215}{\mathbb{N}} % ℕ
\DeclareUnicodeCharacter{2217}{\ensuremath{\ast}} % ∗
\DeclareUnicodeCharacter{2218}{\ensuremath{\circ}} % ∘
\DeclareUnicodeCharacter{2219}{\ensuremath{\bullet}} % ∙ 
\DeclareUnicodeCharacter{2223}{\ensuremath{\mid}} % ∣
\DeclareUnicodeCharacter{2227}{\wedge}% ∧
\DeclareUnicodeCharacter{2228}{\vee}% ∨
\DeclareUnicodeCharacter{2229}{\ensuremath{\cap}} % ∩
\DeclareUnicodeCharacter{222A}{\ensuremath{\cup}} % ∪
\DeclareUnicodeCharacter{2237}{::} % ∷
\DeclareUnicodeCharacter{2243}{\ensuremath{\simeq}} % ≃
\DeclareUnicodeCharacter{2245}{\ensuremath{\cong}} % ≅ 
\DeclareUnicodeCharacter{2248}{\ensuremath{\approx}} % ≈
\DeclareUnicodeCharacter{225C}{\ensuremath{\stackrel{\scriptscriptstyle {\triangle}}{=}}} % ≜
\DeclareUnicodeCharacter{225F}{\ensuremath{\stackrel{\scriptscriptstyle ?}{=}}} % ≟
\DeclareUnicodeCharacter{2260}{\neq}% ≠
\DeclareUnicodeCharacter{2261}{\equiv}% ≡
\DeclareUnicodeCharacter{2264}{\ensuremath{\le}} % ≤
\DeclareUnicodeCharacter{2265}{\ensuremath{\ge}} % ≥
\DeclareUnicodeCharacter{2282}{\ensuremath{\subset}} % ⊂
\DeclareUnicodeCharacter{2283}{\ensuremath{\supset}} % ⊃ 
\DeclareUnicodeCharacter{2286}{\ensuremath{\subseteq}} % ⊆ 
\DeclareUnicodeCharacter{2287}{\ensuremath{\supseteq}} % ⊇
\DeclareUnicodeCharacter{2293}{\ensuremath{\sqcup}} % ⊓
\DeclareUnicodeCharacter{2295}{\ensuremath{\oplus}} % ⊕
\DeclareUnicodeCharacter{2297}{\ensuremath{\otimes}} % ⊗
\DeclareUnicodeCharacter{22A2}{\ensuremath{\vdash}}
\DeclareUnicodeCharacter{22A4}{\ensuremath{\top}} % ⊤
\DeclareUnicodeCharacter{22A5}{\ensuremath{\bot}} % ⊥
\DeclareUnicodeCharacter{22A7}{\models} % ⊧ 
\DeclareUnicodeCharacter{22A8}{\models} % ⊨
\DeclareUnicodeCharacter{22A9}{\Vdash} % ⊩
\DeclareUnicodeCharacter{22B8}{\ensuremath{\multimap}} % ⊸
\DeclareUnicodeCharacter{22C4}{\diamond} % ⋄
\DeclareUnicodeCharacter{25C7}{\diamond} % ◇
\DeclareUnicodeCharacter{22C6}{\ensuremath{\star}}
\DeclareUnicodeCharacter{22EF}{\ensuremath{\cdots}} % ⋯
\DeclareUnicodeCharacter{2308}{\ensuremath{\lceil}}
\DeclareUnicodeCharacter{2309}{\ensuremath{\rceil}}
\DeclareUnicodeCharacter{230A}{\ensuremath{\lfloor}}
\DeclareUnicodeCharacter{230B}{\ensuremath{\rfloor}}
\DeclareUnicodeCharacter{25A1}{\ensuremath{\square}} % □
\DeclareUnicodeCharacter{2605}{\ensuremath{\star}}   % ★
\DeclareUnicodeCharacter{27E6}{\ensuremath{\llbracket}} % ⟦
\DeclareUnicodeCharacter{27E7}{\ensuremath{\rrbracket}} % ⟧
\DeclareUnicodeCharacter{27E8}{\ensuremath{\langle}} % ⟨
\DeclareUnicodeCharacter{27E9}{\ensuremath{\rangle}} % ⟩
\DeclareUnicodeCharacter{27F6}{{\longrightarrow}} % ⟶
\DeclareUnicodeCharacter{27F7}{{\longleftrightarrow}} % ⟷
\DeclareUnicodeCharacter{8499}{\mathcal{M}} 
\DeclareUnicodeCharacter{8718}{\ensuremath{\blacksquare}}
\DeclareUnicodeCharacter{8797}{\mathrel{\mathop:}=}
\DeclareUnicodeCharacter{9657}{\ensuremath{\triangleright}}
\DeclareUnicodeCharacter{9667}{\triangleright{}}
\DeclareUnicodeCharacter{9669}{\ensuremath{\triangleleft}}


\newenvironment{bloc}{\begin{array}[t]{@{}l@{}}}{\end{array}}
\newcommand{\braces}[1]{\left\{ {#1} \right\} }

\newcommand{\p}[1]{{#1}^{⊥}}
\newcommand{\Let}{\mathsf{let}~}
\newcommand{\In}{~\mathsf{in}~}
\newcommand{\Fst}{\mathsf{fst} }
\newcommand{\Snd}{\mathsf{snd} }
\newcommand{\Inl}{\mathsf{inl}~}
\newcommand{\Inr}{\mathsf{inr}~}
\newcommand{\Case}{\mathsf{case}~}
\newcommand{\Of}{\mathsf{~of}~}
\newcommand{\Connect}[2]{\mathsf{connect}~#1~\mathsf{to}~\braces{#2}}
\newcommand{\ConnectS}[5]{\Connect{#1}{#2 \In #3; #4 \In #5}}
\newcommand{\ConnectB}[5]{\Connect{#1}{
    \begin{array}{l@{\In}l}
      #2 & #3 \\ #4 & #5     
    \end{array}
}}
\newcommand{\newchan}[1]{\mathsf{new}_{#1}}
\begin{document}

\maketitle

\def\pat{,}
\def\loll{\multimap}
\def\pa{\ensuremath{\parr}}

\section{Core LL}

\subsection{Syntax}
neutral in parens
\[
\begin{array}{ccc}
\text{conjunction} &   \text{disjuction} & \\
 ⊗ ~ (1) &  \pa ~ (⊥) & \text{multiplicative}  \\
 \& ~ (⊤) &  ⊕ ~ (0) & \text{additive}
\end{array}
\]

\begin{tabular}{cc}
{$A ⊗ B$} & $A$ and $B$ in parallel\\
{$A \pa B$}& $A$ when $\p B$ is given, or vice versa. $A ⊸ B ≜ \p A \pa B$. \\
{$A ⊕ B$}& Either $A$ or $B$ (the context choses which)\\
{$A ~\&~ B$}& Either $A$ or $B$ (the program choses which)\\
\end{tabular}

\subsection{Types}

We have a judgement $Γ ⊢ t$; with only hypotheses and no conclusion. Remarks:
\begin{itemize}
\item In fact, it is as if there were a single $⊥$ conclusion: $Γ ⊢ t
  : ⊥$.  Essentially the programs are written in CPS.  Because the
  return type is always the same, writing it is redundant, so we omit
  it.
\item This is inverted compared to the usual presentation of LL, which
  has only conclusions.
\item As usual in LL, we have only elimination rules; because
  introduction rules are recovered using {\sc Ax} and {\sc Cut}.
\item In sum, when we have an set of hypothesis, if the program is
  ``fed'' inputs for all of them but one, the last one will ``spit
  out'' a result.
\end{itemize}

\newcommand{\chan}[3]{#2 \stackrel{#1}{↔} #3}

\begin{mathpar}
\inferrule[Ax]{ }{ x : \p A, y : A ⊢ x ↔ y}
\and
\inferrule[Cut]{Γ, x:\p A ⊢ a \\ y:A, Δ ⊢ b }{Γ,Δ ⊢ \ConnectS {\newchan A} x a y b}
\\
\inferrule[⊗]{Γ, x : A, y : B, Δ ⊢ a}{Γ, z : A ⊗ B, Δ ⊢ \Let ⟨x,y⟩=z \In a}
\and 
\inferrule[\pa]{Γ, x:A ⊢ a\\ y:B, Δ ⊢ b}{ Γ, z:A\pa B, Δ ⊢ \ConnectS z x a y b}
\\
\begin{array}{c}
\inferrule[\&-l]{Γ,x:A ⊢ a} { Γ,z:A~\&~B ⊢ \Let x = \Fst z \In a}\\
\\ 
\inferrule[\&-r]{Γ,y:B ⊢ b} { Γ,z:A~\&~B ⊢ \Let y = \Snd z \In b}\\
\end{array}
\and
\inferrule[⊕]{Γ,x:A ⊢ a \\ Γ,y:B ⊢ b} { Γ,z:A⊕B ⊢ \Case z \Of \{ \Inl x ↦ a ; \Inr y ↦ b  \} }
\\
\inferrule[1]{Γ ⊢ a}{Γ, x:1 ⊢ \Let ⋄ = x \In a} 
\and
\inferrule[⊥]{ }{x:⊥ ⊢ x} 
\\
\text{no rule for ⊤}
\and
\inferrule[0]{ }{Γ,x:0 ⊢ \mathsf{dump~}Γ\mathsf{~into~} x} 
\\
  \inferrule[∀]{Γ,x:A[B/α] ⊢ b \\ B\text{~valid} }{Γ,z:∀α.A ⊢ \Let x = z∙B \In b }
\and
  \inferrule[∃]{Γ,y:A ⊢ b}{Γ,z:∃α.A ⊢ \Let ⟨α,y⟩ = z \In b }
\end{mathpar}

Explanation of the rules:
\begin{itemize}
\item {\sc Ax} plugs the  $x$ to $y$ (and vice-versa).

  Remark: we do not know from the rule in isolation the ``direction of
  travel'' of information.  Indeed, negation is involutive; so we can
  just reverse the rule by substituting $\p A$ for $A$. 

  However, if $A$ is a monomorphic base type, we can interpret the \emph{instance}
  of the rule to be communication in a specific direction.


\item {\sc Cut}: creates a new communication channel of type $A$;
  naming the ends $x$ and $y$. Programs ``talking'' on either end
  execute in parallel.

  Note that the communication channel is ``one-shot''; but the type of
  the channel may be a list/stream/what have you. There may even be
  back and forth communication if the type involes eg. a linear arrows.

  (iirc, Wadler 2012 makes the opposite choice).

\item {\sc $⊗$}: Essentially no-op. Just gives names to more hypothesis.
\item {\sc \pa}: executes two processes in parallel. Note that the
  only with possible communication is via $z$; the parts $Γ$ and $Δ$
  can be seen as separate memories in the rhs. The situation is
  similar for {\sc Cut}.


\item {\sc ⊕}: does case analysis
\item {\sc \&}: choses a side
\item Because there is no elim rule for ⊤, 0 can never be constructed.
\end{itemize}
TODO: exponentials.

\subsection{Reduction rules (Incomplete)}

\begin{mathpar}
  \ConnectB {\newchan {A ⊗ B}} x {\ConnectS x v a w b} y {\Let ⟨t,u⟩=y \In c} \\ ~⟶~  \ConnectB {\newchan A} v {a} {t} {\ConnectB {\newchan B} w b u c }
\end{mathpar}


\begin{align*}
\ConnectB {\newchan A} x {x ↔ z} y t & ~⟶~ t[z/x] \\ 
\ConnectB {\newchan {A⊕B}} x {\Let z = \Fst x \In a} y {\Case y \Of \{ \Inl t ↦ b ; \Inr u ↦ c  \}} & ~⟶~ \ConnectB {\newchan A} z a t b \\
\ConnectB {\newchan {}} {x:∀α.A} {\Let t = x ∙ B \In a} {y:∃α.\p A} {\Let ⟨α,u⟩ = y \In b }  & ~⟶~ \ConnectB {\newchan {}} {t:A[B/α]} {a} {u:\p{A[B/α]}} {b} 
\end{align*}

\subsection{Virtual machine}
\newcommand\layout[1]{[#1]}

The idea of the VM is to delay cut-elimination in order to get a more
efficient execution (as in eg. Krivine's abstract machine). Concurrent
semantics naturally arise.

The VM reduces/executes a number of closures concurrently. (Each closure can be thought of as a process.)
A closure is a sequent together with an environment. An environment is, for each variable $x:A$ in the context:
\begin{itemize}
\item a memory area of layout $\layout A$ which can recieve data for the variable/channel. 
\item a pointer to the converse memory area (of layout $\layout {\p
    A}$) in another closure's environment.
\end{itemize} (Each variable can be thought of as a channel.)  Note
that we will execute closed programs only, so there is no ``open''
environment: each variable is connected.

The translation of each type to a memory layout is: ($·$ indicates sequence, $|$ is overlay.)
\begin{align*}
  \layout{A⊗B} &= \layout A · \layout B\\
  \layout{A \pa B} &= \layout A · \layout B\\
  \layout{A ⊕ B} &= \text{1 bit of info + 1 bit indicating that the data is ready} · (\layout A \mid \layout B)\\
  \layout{A \& B} &= \layout A \mid \layout B \\
  \layout{α} &=  ? \\
  \layout{!A} &= \text{pointer to}~\layout{A} \\
  \layout{T} &= \epsilon & \text{for neutrals}
\end{align*}

Operational behaviour of the rules:
\begin{itemize}
\item 0: crash
\item 1: continue
\item ⊥: terminate (delete the closure)
\item ⊕: wait for the data to be ready; then chose a branch according
  to the bit of data. (The memory for the 1 bit + semaphore could be
  freed here, but probably this should wait for a weaken rule to be
  run).
\item \&: write a bit of data (which becomes ready), and continue.
\item $x:(A ⊗ B)$: add an entry in the context for $y:B$, at location
  $x+\layout A$ (No communication occurs!)
\item $A \pa B$: split the environment and spawn a new closure. (No communication either)
\item Cut: similar to $\pa$ but connect the two closures together.
\item Ax: Fix the pointers in the connected closures, so that they
  point to each other. Then terminate.
\end{itemize}

\subsection{Example}
Let us prove/program $A ⊗ B ⊸ B ⊗ A$. Putting everything as
hypotheses, we must find some $p$ with
%
\[x : A ⊗ B, y : \p {(B ⊗ A)} ⊢ p\]
%
Expanding the definition of negation:

\begin{align*}
  x : A ⊗ B, y : \p B \pa \p A ⊢ p\\
  p & = \Let ⟨v:A,w:B⟩ = x \In  \\
    & \quad \ConnectB y {t:\p B} {t ↔ w} {u : \p A} {u ↔ v}
\end{align*}


\section{Cut and update the world}
Can linear types change the world?

Assume a type $W$ containing the state of the world. Two sequents can
be composed by Cut:

\AXC {$W,\p W, \p X ⊢ $}
\AXC {$W,\p W, \p Y⊢ $ }
\LL{\sc Cut $W$}\BIC{$W,\p W, \p X, \p Y⊢ $}
\UIC{$W,\p W, {(\p X ⊗ \p Y)}⊢ $}
\UIC{$W,\p W, \p {(X \pa Y)}⊢ $}
\DisplayProof

Maybe if $W$ is abstract enough it's possible to update it in place.

\section{Functional Syntax}
(fragments; see DILL for the rest)
\begin{mathpar}
\inferrule{Γ, x:\p A ⊢ a}{Γ ⊢ \mathsf{return} ~x:\p A~ \mathsf{from} ~a : A}
\and \inferrule{Γ ⊢ a : A \\ Δ ⊢ b}{Γ,Δ,x:\p A ⊢ \Let x = a \In b}
\\
\and \inferrule{Γ ⊢ a : A \\ Δ ⊢ b : B}{Γ,Δ ⊢ ⟨a,b⟩ : A⊗B}
\\
% ??? \and \inferrule{Γ ⊢ f : A ⊸ B \\ Δ ⊢ a : B}{Γ ⊢ f a : B}
\and \inferrule{Γ,x:A ⊢ b:B}{Γ ⊢ λx:A.b : A ⊸ B}
\\
\and \inferrule{Γ ⊢ a : A}{Γ ⊢ \Inl a : A ⊕ B}
 
\end{mathpar}


\begin{mathpar}
\end{mathpar}



\end{document}


